\chapter*{Introduction générale}
%\section{Introduction}
\tab Avec le développement logiciel continu et rapide, un programme subit beaucoup de changements pendant son cycle de vie, ce qui engendre des fautes suite à des mauvaises habitudes exercées par les développeurs sous préssion ou par ignorance. Ces mauvaises pratiques affecte la qualité logicielle et rend difficile la compréhesion du code que ce soit par l’auteur du code lui même aprés quelques modifications ou bien par d’autres développeurs lors de la maintenance. Cette derniere avec ses types: Corrective, prenventive, adaptive et  perfective, represente une activité enivitable lors du développement logiciel et vise à produire un logiciel de qualité. \\ \tab

Pour améliorer la qualité logicielle, plusieurs techniques existent, parmi ces techniques: l’utilisation de l’information linguistique, l’utilisation des bons identificateurs, la bonne documentation et l’utilisation du bon lexique. C’est pourquoi, de nouveaux travaux de recherches sont apparus pour exploiter l’information linguistique d’un programme pour identifier les mauvaises pratiques liées au coté linguistique d’un programme ce qui a mené à l’apparition des «anti-patrons linguistiques». \\ \tab

L’objectif de ce rapport est de comprendre ce qu’est un anti-patron linguistique, les anti-patrons linguistiques connus jusqu’à maintenant et comment les détecter.