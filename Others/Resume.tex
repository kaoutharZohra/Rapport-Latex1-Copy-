% Chapter Template
\chapter*{Résumé}% Main chapter title
\addtotoc{Résumé}
\label{Chapter2} % Change X to a consecutive number; for referencing this chapter elsewhere, use \ref{ChapterX}

\lhead{\emph{Détection de personnes et navigation sociale d’un robot guide}}
\setstretch{1.3}
\tab Actuellement, nous vivons un grand progrès des nouvelles technologies, particulièrement dans la robotique mobile et la robotique de service. Les robots mobiles ont envahi l'espace humain, ils le partagent avec eux, ce qui les oblige à gérer leurs interactions avec les personnes d'une manière à s'intégrer dans la société humaine. Pour accomplir les différentes tâches ou les différents services qu'offre le robot social à l'homme, la robotique mobile fait appel à différents domaines de recherche. L'un des domaines les plus exploités dans la robotique de service est la perception, particulièrement la détection des objets moyennant différents capteurs. La nature du robot social lui impose une interaction avec les humains par conséquent, il parvient un fort besoin de les détecter et de les distinguer des autres objets. Les robots sociaux interviennent dans des milieux, généralement, peuplés ce qui leurs exige le respect de l'aspect social des humains et l'adhérence à un ensemble de conventions sociales.\vspace{5px}\\
\tab Le présent rapport consiste à la conception et à la mise en œuvre d'une détection de personnes et d'une navigation sociale pour un robot guide. En exploitant des algorithmes de détection du visage et du squelette qui s'opèrent sur les données du capteur de vision Kinect et d'autres détecteurs de jambes qui utilisent les données du capteur laser, nous présentons une conception qui améliore le module de détection de personnes pour un robot guide. Une fois le module de détection de personnes fournit des résultats satisfaisants, nous les exploitons dans la navigation sociale du robot en veillant à respecter un ensemble de règles et de conventions sociales. Ces conventions sont inspirées de la théorie de proxémie et du partage de l'espace social autour d'une personne. Notre système est testé sur un simulateur qui fait preuve de son bon fonctionnement, il est ensuite vérifié sur la plate-forme robotique réelle qui confirme les résultats de la simulation.\vspace{5px}\\
\tab \textbf{Les mots clés : } la robotique mobile, la robotique de service, interaction, robot social, la détection des objets, capteurs, Kinect, laser, détection de personnes, navigation sociale, règles et conventions sociales, la théorie de proxémie.

