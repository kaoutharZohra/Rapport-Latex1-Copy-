% Chapter Template



\begin{otherlanguage}{arabic}

\chapter*{ \AR {ملخص}}
\addtotoc{\AR {ملخص}}
\label{Chapter4} 
\lhead{\emph{\AR{ملخص}}}
\tab في وقتنا الحالي, نحن نعيش في تطور كبير للتكنولوجيات الحديثة، ولا سيما في مجال الروبوتات المتنقلة و روبوتات الخدمة. فقد غزت الروبوتات المتحركة الفضاء البشري فأصبحوا يشاركونه معهم، مما أدى إلى إجبارهم على إدارة تفاعلاتهم مع الناس  للإندماج في المجتمع البشري. ولإنجاز المهام أو الخدمات المختلفة التي يقدمها الروبوت الاجتماعي للإنسان، تقوم الروبوتات المتنقلة استغلال مجالات بحث المختلفة. من أهم هذه المجالات الموظفة في مجال روبوتات الخدمة هو إدراك المحيط الخارجي،  خاصة الكشف عن الأجسام من خلال أجهزة استشعار مختلفة. إن طبيعة الروبوت الاجتماعي تفرض عليه التفاعل مع البشر، لذلك، هناك حاجة شديدة للكشف عن الأشخاص وتمييزها عن غيرها من الأشياء. فهو يعمل في بيئات  عادة ما تكون مكتظة بالسكان مما يوجبه على احترام مطالب الجانب الاجتماعي للإنسان و إتباع مجموعة من التقاليد الاجتماعية.\vspace{5px}\\
\tab في هذا التقرير نعرض تصميم وتنفيذ مشروع  يقوم بالكشف عن الأشخاص والتنقل الاجتماعي لروبوت موجه. عن طريق خوارزميات للكشف عن الوجه والهيكل العظمي التي تعمل على البيانات المستخرجة من أجهزة استشعار الرؤية \textLR{Kinect} وأيضا أجهزة استشعار أخرى للساقين التي تستخدم بيانات من جهاز استشعار الليزر، التصميم الذي قمنا به يحسن وحدة الكشف عن الأشخاص لدى روبوت الموجه. فعندما توفر وحدة الكشف عن الأشخاص نتائج مرضية، نقوم  بإستغلالها في التنقل الاجتماعي للروبوت مع مراعاة مجموعة من القواعد والأعراف الاجتماعية المستوحاة من نظرية مبحث التداني وتقاسم الفضاء الاجتماعي حول شخص ما. يتم اختبار النظام على برنامج محاكاة الذي يوضح العملية ، ثم يتم التحقق من ذلك على المنصة الروبوتية الفعلية مما يؤكد نتائج المحاكاة.\vspace{5px}\\
\tab \textbf{الكلمات المفتاحية : }الروبوتات المتنقلة، روبوتات الخدمة، تفاعلات، الروبوت الاجتماعي، إدراك المحيط ، الكشف عن الأجسام ، أجهزة استشعار، الجانب الاجتماعي للإنسان ، التقاليد الاجتماعية ، الكشف عن الأشخاص، \textLR{Kinect} ، الليزر، التنقل الاجتماعي ،  نظرية مبحث التداني.
	
\end{otherlanguage}