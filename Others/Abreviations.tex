% Chapter Template
\chapter*{Glossaire}% Main chapter title
\addtotoc{Glossaire}
\label{Chapter5} % Change X to a consecutive number; for referencing this chapter elsewhere, use \ref{ChapterX}

\lhead{\emph{Glossaire}}

\textbf{2D :} Deux dimensions.\vspace{5px}\\
\textbf{3D :} Trois dimensions.\vspace{5px}\\
\textbf{B21r :} Robot mobile de type iRobot.\vspace{5px}\\
\textbf{GUI :} Graphical User Interface.\vspace{5px}\\
\textbf{Laser : }light amplification by stimulated emission of radiation.\vspace{5px}\\
\textbf{OpenCV :} Une bibliothèque pour le traitement d’images. \vspace{5px}\\
\textbf{PC :} Personal Computer.\vspace{5px}\\
\textbf{SDF :} Simulation Description Format. \vspace{5px}\\
\textbf{SLAM :} Simultaneous Localization And Mapping (Localisation simultanée et cartographie).\vspace{5px}\\
\textbf{RAM :} Random Access Memory.\vspace{5px}\\
\textbf{RGB :} Red, Green, Blue (rouge, vert, bleu).\vspace{5px}\\
\textbf{ROS :} Robot Operating System (Système d’exploitation du robot).\vspace{5px}\\
\textbf{TF :} Transform Frame.\vspace{5px}\\
\textbf{UML :} Unified Modeling Language. \vspace{5px}\\
\textbf{URDF :} Universal Robotic Description Format \vspace{5px}\\
\textbf{XACRO :} XML Macro. \vspace{5px}\\
\textbf{XML :}  Extensible Markup Language \vspace{5px}\\


